\documentclass[11pt,a4paper]{article}

\usepackage[margin=0.3in]{geometry}
\usepackage[utf8]{inputenc}
\usepackage[english]{babel}
\usepackage[T1]{fontenc}
\usepackage{lmodern}
\usepackage{amsmath}
\usepackage{amssymb}

\def\ra{\rightarrow}
\def\cc{\,\|\,}
\def\gc{\,\square\,}
\def\ch{\,|\,}
\def\ic{\,\sqcap\,}
\def\co{\backslash}
\def\cat{^{\frown}}
\def\filter{\,\upharpoonright\,}
\newcommand{\ab}[1]{\left \langle #1 \right \rangle}
\newcommand{\sN}[1]{\left \lbrace #1 \right \rbrace}

\renewcommand{\thesubsection}{\thesection.\alph{subsection}}

\title{Extreme Multiprogramming\\Assignment 1 - theory}
\author{Malte Stær Nissen}

\begin{document}
\maketitle

\section{Satisfiability proof}



\section{Derrivation of process definitions}
\subsection{$a = V \cc R$}

\begin{align*}
    \alpha \left( Sa \right) &= \alpha(V)
    \cup \alpha(R) = \sN{c,p,s} \cup \sN{c,t} = \sN{c,p,s,t} \\
    Sa &= c \ra (\underbrace{\left( p \ra V \ch s \ra v \right)}_{Sa'}
     \cc \underbrace{\left( c \ra t \ra R \right)}_{R'}) & [2.2.1,~L4A] \\
    A &= \sN{p,c},~~B = \sN{t},~~C = \sN{p,t} & [2.3.1,~L7] \\
    Sa &= c \ra \left(p \ra  (V \cc R') \ch t \ra (R \cc V') \right) \\
       &= c \ra \left(p \ra (t \ra (V \cc R))
                  \ch t \ra \left( c \ra t \ra R\right) \cc \left(p \ra V
                         \ch s \ra v \right) \right) & [2.3.1,~L5B] \\
       &= c \ra \left( p \ra ( t \ra Sa)
                  \ch t \ra ( \left( c \ra t \ra R\right) \cc \left(p \ra V
                         \ch s \ra v \right) ) \right) & [2.3.1,~L5B] \\
    A &= \sN{c},~~B=\sN{p,c},~~C=\sN{c,p} & [2.3.1,~L7] \\
    Sa &= c \ra \left( p \ra (t \ra Sa)
                  \ch t \ra ( p \ra (V \cc R) \ch c \ra( (t \ra R) \cc (s \to V)))
                  \right) \\
      &= c \ra \left( p \ra (t \ra Sa)
                  \ch t \ra ( p \ra Sa \ch c \ra (t \ra (R \cc (s \to V))
                  \ch s \ra ( (t \ra R) \cc V)))
                  \right) & [2.3.1,~L6] \\
       &= c \ra \left( p \ra (t \ra Sa)
                  \ch t \ra ( p \ra Sa \ch c \ra (t \ra (s \ra (R \cc V))
                  \ch s \ra ( (t \ra R) \cc V)))
                  \right) & [2.3.1,~L5B] \\
       &= c \ra \left( p \ra (t \ra Sa)
                  \ch t \ra ( p \ra Sa \ch c \ra (t \ra (s \ra Sa)
                  \ch s \ra ( t \ra (R \cc V))))
                  \right) & [2.3.1,~L5A] \\
       &= c \ra \left( p \ra (t \ra Sa)
                  \ch t \ra ( p \ra Sa \ch c \ra (t \ra (s \ra Sa)
                  \ch s \ra (t \ra Sa)))
                  \right)
\end{align*}

\subsection{$Sb = Sa \cc T1$}

\begin{align*}
    \alpha \left( Sb \right) &= \alpha(Sa) \cup \alpha(T1) = \sN{c,p,s,t} \\
    sb &= (c \ra \left( p \ra (t \ra sa)
                  \ch t \ra ( p \ra sa \ch c \ra (t \ra (s \ra sa)
                  \ch s \ra (t \ra sa)))
                  \right)) \cc (s \ra t1) & \\
       &= c \ra (\left( p \ra (t \ra sa)
                  \ch t \ra ( p \ra sa \ch c \ra (t \ra (s \ra sa)
                  \ch s \ra (t \ra sa)))
                  \right) \cc (s \ra t1)) & [2.3.1,~L5A] \\
       A &=\sN{p,t},~~B=\sN{s},~~C=\sN{t} & [2.3.1,~L7] \\
    Sb &= c \ra t \ra ( ( p \ra Sa \ch c \ra (t \ra (s \ra Sa)
                  \ch s \ra (t \ra Sa)))
                   \cc (s \ra T1)) & \\
     A &=\sN{p,c},~~B=\sN{s},~~C=\sN{c} & [2.3.1,~L7] \\
    Sb &= c \ra t \ra c \ra ((t \ra (s \ra Sa)
                  \ch s \ra (t \ra Sa))
                   \cc (s \ra T1)) & \\
     A &=\sN{t,s},~~B=\sN{s},~~C=\sN{s,t} & [2.3.1,~L7] \\
    Sb &= c \ra t \ra c \ra (t \ra ((s \ra Sa) \cc (s \ra T1))
                  \ch s \ra ((t \ra Sa) \cc T1)) & \\
       &= c \ra t \ra c \ra (t \ra (s \ra Sa \cc T1)
                  \ch s \ra ((t \ra Sa) \cc T1)) & [2.3.1,~L4A] \\
       &= c \ra t \ra c \ra (t \ra (s \ra Sb)
                  \ch s \ra (t \ra Sa \cc T1)) & [2.3.1,~L5A] \\
       &= c \ra t \ra c \ra (t \ra (s \ra Sb)
                  \ch s \ra (t \ra Sb)) & \\
\end{align*}

\subsection{$Sc = R \cc T1$}

\begin{align*}
    \alpha(Sc) &= \alpha (R) \cup \alpha(T1) = \sN{c,t} \cup \sN{s,p} = \sN{c,t,s,p} & \\
    Sc &= (c \ra t \ra R) \cc (s \ra T1) & \\
       &= (c \ra ((t \ra R)\cc (s \ra T1)) \ch s \ra ( (c \ra (t \ra R)) \cc T1)) & [2.3.1,~L6] \\
       &= (c \ra ((t \ra R)\cc (s \ra T1)) \ch s \ra Sc) & \\
    Sc' &= (t \ra R) \cc T1 & \\
        &= (t \ra R) \cc (s \ra T1) \\
    &= (t \ra ( R \cc (s \ra T1)) \ch s \ra ((t \ra R) \cc T1)) & [2.3.1,~L6] \\
    &= (t \ra Sc \ch s \ra Sc') & \\
    Sc &= (c \ra ((t \ra R)\cc (s \ra T1)) \ch s \ra  Sc) & \\
       &= (c \ra (t \ra (R \cc (s \ra T1)) \ch s \ra ((t \ra R) \cc T1)) \ch s \ra  Sc) & [2.3.1,~L6] \\
       &= (c \ra (t \ra Sc \ch s \ra Sc') \ch s \ra  Sc) & \\
\end{align*}

\subsection{$Sd = V \cc Sc$}

\begin{align*}
    \alpha(Sd) &= \alpha(V) \cup \alpha(Sc) = \sN{c,p,s} \cup \sN{c,t,s,p} =
    \sN{c,t,s,p} & \\
    Sd &= (c \ra (p \ra V \ch c \ra s \ra V ) ) \cc
          (c \ra (t \ra Sc \ch s \ra Sc') \ch s \ra Sc) & \\
     A &= \sN{c},~~ B = \sN{c,s},~~ C = \sN{c} & [2.3.1,~L7] \\
    Sd &= c \ra (( p \ra V \ch c \ra s \ra V) \cc (t \ra Sc \ch s \ra Sc')) & \\
    A &= \sN{p,c},~~ B = \sN{t,s},~~C = \sN{t} & [2.3.1,~L7] \\
       &= c \ra t \ra (( p \ra V \ch c \ra s \ra V) \cc Sc) & \\
       &= c \ra t \ra (( p \ra V \ch c \ra s \ra V) \cc (c \ra (t \ra Sc \ch s \ra Sc') \ch s \ra  Sc)) & \\
     A &= \sN{p,c},~~ B = \sN{c,s},~~ C = \sN{c} & [2.3.1,~L7] \\
    Sd &= c \ra t \ra c \ra ((s \ra V) \cc (t \ra Sc \ch s \ra Sc')) & \\
     A &= \sN{s},~~ B = \sN{t,s},~~ C = \sN{s,t} & [2.3.1,~L7] \\
    Sd &= c \ra t \ra c \ra (s \ra (V \cc Sc') \ch t \ra ((s \ra V) \cc Sc)) & \\
       &= c \ra t \ra c \ra
            (s \ra ((c \ra (p \ra V \ch c \ra s \ra V)) \cc
            (t \ra Sc \ch s \ra Sc')) \ch t \ra ((s \ra V) \cc Sc)) & \\
     A &= \sN{c},~~ B = \sN{t,s},~~ C = \sN{t} & [2.3.1,~L7] \\
    Sd &= c \ra t \ra c \ra
            (s \ra (t \ra ((c \ra (p \ra V \ch c \ra s \ra V))) \cc
            Sc) \ch t \ra ((s \ra V) \cc Sc))) & \\
       &= c \ra t \ra c \ra
            (s \ra (t \ra (V \cc Sc)) \ch t \ra ((s \ra V) \cc
            (c \ra (t \ra Sc \ch s \ra Sc')) \ch s \ra Sc)) & \\
     A &= \sN{s},~~ B = \sN{c,s},~~ C = \sN{s} & [2.3.1,~L7] \\
    Sd &= c \ra t \ra c \ra
            (s \ra (t \ra Sd) \ch t \ra (s \ra (V \cc Sc))) & \\
       &= c \ra t \ra c \ra
            (s \ra (t \ra Sd) \ch t \ra (s \ra Sd)) & \\
\end{align*}

\subsection{$Se = Sa \cc T2$}

\begin{align*}
    \alpha(Se) &= \alpha(Sa) \cup \alpha(T2) = \sN{c,p,s,t} \cup \sN{s,p} =
    \sN{c,p,s,t} & \\
    Se &= (c \ra \left( p \ra (t \ra Sa)
                  \ch t \ra ( p \ra Sa \ch c \ra (t \ra (s \ra Sa)
                  \ch s \ra (t \ra Sa)))
                  \right)) \cc (p \ra T2) \\
       &= c \ra (\left( p \ra (t \ra Sa)
                  \ch t \ra ( p \ra Sa \ch c \ra (t \ra (s \ra Sa)
                  \ch s \ra (t \ra Sa)))
                  \right) \cc (p \ra T2)) & [2.3.1,~L5A] \\
     A &= \sN{p,t},~~ B = \sN{p},~~ C = \sN{p,t} & [2.3.1,~L7] \\
    Se &= c \ra \left( p \ra ((t \ra Sa) \cc T2)
                  \ch t \ra (( p \ra Sa \ch c \ra (t \ra (s \ra Sa)
                      \ch s \ra (t \ra Sa))) \cc (p \ra T2))
                      \right) & \\
       &= c \ra \left( p \ra (t \ra (Sa \cc T2))
                  \ch t \ra (( p \ra Sa \ch c \ra (t \ra (s \ra Sa)
                      \ch s \ra (t \ra Sa))) \cc (p \ra T2))
                      \right) & [2.3.1,~L5A] \\
       &= c \ra \left( p \ra (t \ra Se)
                  \ch t \ra (( p \ra Sa \ch c \ra (t \ra (s \ra Sa)
                      \ch s \ra (t \ra Sa))) \cc (p \ra T2))
                      \right) & \\
     A &= \sN{p,c},~~ B = \sN{p},~~ C = \sN{p,c} & [2.3.1,~L7] \\
    Se &= c \ra \left( p \ra (t \ra Se)
                  \ch t \ra ( p \ra (Sa \cc T2) \ch c \ra ((t \ra (s \ra Sa)
                      \ch s \ra (t \ra Sa) \cc (p \ra T2))))
                      \right) & \\
       &= c \ra \left( p \ra (t \ra Se)
                  \ch t \ra ( p \ra Se \ch c \ra ((t \ra (s \ra Sa)
                      \ch s \ra (t \ra Sa) \cc (p \ra T2))))
                      \right) & \\
     A &= \sN{t,s},~~ B = \sN{p},~~ C = \sN{t} & [2.3.1,~L7] \\
    Se &= c \ra \left( p \ra (t \ra Se)
                  \ch t \ra ( p \ra Se \ch c \ra (t \ra ((s \ra Sa) \cc (p \ra
                  T2))))
                      \right) & \\
       &= c \ra \left( p \ra (t \ra Se)
                  \ch t \ra ( p \ra Se \ch c \ra (t \ra STOP))
                  \right) & [2.3.1,~L4B] \\
\end{align*}

\subsection{$Sf = V \cc f(Sc)$}

\begin{align*}
    \alpha(Sf) &= \alpha(V) \cup \alpha(f(Sc)) = \sN{c,p,s} \cup \sN{c,t,s,p} =
    \sN{c,t,s,p} & \\
    f(c) &= c,~~  f(t) = t,~~ f(p) = s,~~ f(s) = p & \\
    f(Sc) &= f(c \ra (t \ra Sc \ch s \ra Sc') \ch s \ra  Sc) & \\
          &= (c \ra (t \ra f(Sc) \ch p \ra f(Sc')) \ch p \ra f(Sc)) & [2.6.1,~L2] \\
    f(Sc') &= f(t \ra Sc \ch s \ra Sc') & \\
           &= (t \ra f(Sc) \ch p \ra f(Sc')) & [2.6.1,~L2] \\
    Sf &= (c \ra (p \ra V \ch c \ra s \ra V)) \cc
            (c \ra (t \ra f(Sc) \ch p \ra f(Sc')) \ch p \ra f(Sc)) & \\
     A &= \sN{c},~~ B = \sN{c,p},~~ C = \sN{c} & [2.3.1,~L7] \\
    Sf &= c \ra ((p \ra V \ch c \ra s \ra V) \cc
                  (t \ra f(Sc) \ch p \ra f(Sc'))) & \\
     A &= \sN{p,c},~~ B = \sN{t,p},~~ C = \sN{p,t} & [2.3.1,~L7] \\
    Sf &= c \ra (p \ra (V \cc f(Sc'))
             \ch t \ra ( (p \ra V \ch c \ra s \ra V) \cc f(Sc)) ) & \\
    V \cc f(Sc') &= (c \ra (p \ra V \ch c \ra s \ra V))
                \cc (t \ra f(Sc) \ch p \ra f(Sc')) & \\
     A &= \sN{c},~~ B = \sN{t,p},~~ C = \sN{t} & [2.3.1,~L7] \\
    V \cc f(Sc') &= t \ra ((c \ra (p \ra V \ch c \ra s \ra V)) \cc f(Sc)) & \\
                 &= t \ra (V \cc f(Sc)) = t \ra Sf & \\
    Sf &= c \ra (p \ra (t \ra Sf)
    \ch t \ra \underbrace{( (p \ra V \ch c \ra s \ra V) \cc f(Sc))}_{Sf'} ) & \\
    Sf'   &= (p \ra V \ch c \ra s \ra V) \cc f(Sc) & \\
          &= (p \ra V \ch c \ra s \ra V) \cc
             (c \ra (t \ra f(Sc) \ch p \ra f(Sc')) \ch p \ra f(Sc)) & \\
        A &= \sN{p,c},~~ B = \sN{c,p}, C = \sN{p,c} & [2.3.1,~L7] \\
    Sf'   &= (p \ra (V \cc f(Sc))  \ch c \ra ((s \ra V) \cc
             (t \ra f(Sc) \ch p \ra f(Sc)))) & \\
          &= (p \ra Sf  \ch c \ra ((s \ra V) \cc
             (t \ra f(Sc) \ch p \ra f(Sc)))) & \\
        A &= \sN{s},~~ B = \sN{t,p},~~ C = \sN{t} & [2.3.1,~L7] \\
    Sf'   &= (p \ra Sf  \ch
              c \ra (t \ra ((s \ra V) \cc f(Sc)))) & \\
          &= (p \ra Sf  \ch
              c \ra (t \ra ((s \ra V) \cc
                            (c \ra (t \ra f(Sc) \ch p \ra f(Sc')) \ch p \ra f(Sc))))) & \\
        A &= \sN{s},~~ B = \sN{c,p},~~ C = \sN{ } & [2.3.1,~L7] \\
    Sf'   &= (p \ra Sf \ch c \ra (t \ra STOP)) & \\
    Sf &= c \ra (p \ra (t \ra Sf)
            \ch t \ra  (p \ra Sf \ch c \ra (t \ra STOP))) & \\
\end{align*}

\section{Processes and deadlocking}

\subsection{Non-deadlocking processes}
I am in this quested asked to argue that none of the processes from 2(a-d) can
deadlock.

For 2a it's quite simple to prove by using [2.7,~L2]. None of the
processes $V$ and $R$ stop themselves and $\alpha(V) \cap \alpha(R) = \sN{c}$ so
all conditions for the law stated are fulfilled and hence $Sa = (V \cc R)$ never
stops.

For the rest of the processes 1(b-d) I will adhere from proving formally that a
deadlock will not occur. If we instead just look at the final definitions with
no concurrent compositions, we realize that deadlocking is not appearing. All
the definitions are recursively defined and none of the definitions contain any
$STOP$s and hence deadlocks will not occur unless we introduce new concurrent
processes in which case the processes are no longer defined as in 2(a-d).

\subsection{Deadlock of 2(e)}

We need to find a trace for which the process from 2(e) deadlocks. By looking at
the process definition we obviously need the following trace $s = \ab{c, t, c,
t}$, which I will now formally derrive from $Se$ using the trace laws. In order
to shorten the prooflines I have chosen to use rules [1.8.3,~L3A] and
[1.8.3,~L4] to denote $Se$ after a trace.

\begin{align*}
    Se &= c \ra \left( p \ra (t \ra Se)
                  \ch t \ra ( p \ra Se \ch c \ra (t \ra STOP))
                  \right) & \\
    traces(Se) &= \sN{\ab{ }} \cup \sN{\ab{c} \cat t \ch
                      t \in traces(Se / \ab{c})} & [1.8.1,~L2] \\
    traces(Se/\ab{c}) &= \sN{t \ch t = \ab{ }
        \lor (t_0 = p \land t' \in traces(Se/\ab{c,p}))
        \lor (t_0 = t \land t' \in traces(Se/\ab{c,t})) }  & [1.8.1,~L3] \\
        &= \sN{\ab{ } } \cup \sN{\ab{p} \cat t \ch t \in traces(Se/\ab{c,p})}
                        \cup \sN{\ab{t} \cat t \ch t \in traces(Se/\ab{c,t})} & \\
    traces(Se/\ab{c,t}) &= \sN{\ab{ }} \cup
                           \sN{\ab{p} \cat t \ch t \in traces(Se/\ab{c,t,p})} \cup
                           \sN{\ab{c} \cat t \ch t \in traces(Se/\ab{c,t,c}}
                           & [1.8.1,~L3] \\
    traces(Se/\ab{c,t,c}) &= \sN{\ab{ }} \cup
                             \sN{\ab{t} \cat t \ch t \in
                             traces(Se/\ab{c,t,p,t})} & [1.8.1,~L2] \\
    traces(Se/\ab{c,t,c,t}) &= \sN{\ab{ } } & [1.8.1,~L1] \\
    traces(Se) &\supseteq \sN{\ab{ }, \ab{c,p}, \ab{c,t}, \ab{c,t,p}, \ab{c,t,c},
    \ab{c,t,c,t}} & \\
    s &= \ab{c,t,c,t} \in traces(Se) & \\
    Se/s &= STOP & 4\times[1.8.1,~L3A]
\end{align*}

\subsection{$Sx = (V \cc R) \cc T2$}

We need to show that $s = \ab{c,t,c,t} \in traces(Sx)$. First we define the
trace of $Sx$:
\begin{align*}
    Sx &= (\underbrace{V \cc R}_{Sx'}) \cc T2 & \\
    traces(Sx') &= \sN{t \ch (t \filter \alpha V) \in traces(V) \land
                            (t \filter \alpha R) \in traces(R) \land
                             t \in (\alpha V \cup \alpha R)^*} & [2.3.3,~L1] \\
    traces(Sx) &= \sN{t \ch (t \filter \alpha Sx') \in traces(Sx') \land
                            (t \filter \alpha T2) \in traces(T2) \land
                             t \in (\alpha Sx' \cup \alpha T2)^*} & [2.3.3,~L1] \\
    s \filter \alpha V &= \ab{c,c} & [1.6.2,~L1-L4] \\
    s \filter \alpha R &= \ab{c,t,c,t} & [1.6.2,~L1-L4] \\
    s \filter \alpha Sx' &= \ab{c,t,c,t} & [1.6.2,~L1-L4] \\
    s \filter \alpha T2 &= \ab{ } & [1.6.2,~L1-L4]
\end{align*}
The four restricted traces are computed using the restriction laws and the
details are left out since I found them too simple and cumbersome to describe in
details.

Proving that these last traces are part of their respective processes is enough
to show that $s \in tractes(Sx)$. Once again we will use the shorthand notation
$V / s'$ to denote the process $V$ after a trace $s' \in traces(V)$.
\begin{align*}
    traces(V) &= \sN{\ab{ } } \cup \sN{\ab{c} \cat t \ch t \in traces(V /
    \ab{c})} & [1.8.1,~L2] \\
    traces(V/\ab{c}) &= \sN{\ab{ }}
        \cup \sN{\ab{p} \cat t \ch t \in traces(V / \ab{c,p})}
        \cup \sN{\ab{c} \cat t \ch t \in traces(V / \ab{c,c})} & [1.8.1,~L3] \\
    traces(V) &\supseteq \sN{\ab{ }, \ab{c}, \ab{c,p}, \ab{c,c}} \\
    (s \filter \alpha V) &\in traces(V)
\end{align*}

\begin{align*}
    traces(R) &= \sN{\ab{ }}
        \cup \sN{\ab{c} \cat t \ch traces(R/\ab{c})} & [1.8.1,~L2] \\
    traces(R/\ab{c}) &= \sN{\ab{ }}
        \cup \sN{\ab{t} \cat t \ch traces(R/\ab{c,t})} & [1.8.1,~L2] \\
    traces(R/\ab{c,t}) &= \sN{\ab{ }}
        \cup \sN{\ab{c} \cat t \ch traces(R/\ab{c,t,c})} & [1.8.1,~L2] \\
    traces(R/\ab{c,t,c}) &= \sN{\ab{ }}
        \cup \sN{\ab{t} \cat t \ch traces(R/\ab{c,t,c,t})} & [1.8.1,~L2] \\
    traces(R) &\supseteq \sN{\ab{ }, \ab{c}, \ab{c,t}, \ab{c,t,c}, \ab{c,t,c,t}}
    \\
    (s \filter \alpha R) &\in traces(R)
\end{align*}
Furthermore the statement $(s \filter \alpha Sx') \in (\alpha V \cup \alpha R)^*$ holds since $(\alpha
V \cup \alpha R) = \alpha Sa = \sN{c,p,s,t}$ and hence we have shown that
$s \in traces(Sx')$. Furthermore $s \in (\alpha Sx' \cup \alpha T2)^*$ since
$(\alpha Sx' \cup \alpha T2) = \alpha Se = \sN{c,p,s,t}$.

Since $Sx$ is defined using $Sx'$, all which is left to show is
that $(s \filter \alpha T2) \in traces(T2)$ holds. This is however trivial since
$s \filter \alpha T2 = \ab{}$. The empty trace is always in the trace of
a process by [1.8.1,~L6] and hence we have shown that $s \in traces(Sx)$.

\subsection{$Sy = Sx/s$}
We first compute $Sy = Sx/s$:
\begin{align*}
    Sx &= (V \cc R) \cc T2 & \\
    Sy &= Sx/s = ((V/(s \filter \alpha V))
                \cc (R/(s \filter \alpha R))
                \cc (T2/(s \filter \alpha T2)))
                & 2\times[2.3.3,~L2] \\
    &s \filter \alpha V = \ab{c,c},
    ~~ s \filter \alpha R = \ab{c,t,c,t},
    ~~ s \filter \alpha R2 = \ab{ } & \text{(as in 3c)} \\
    V/\ab{c,c} &= (V / \ab{c}) / \ab{c} & [1.8.3,~L2] \\
               &= s \ra VM & 2 \times [1.8.3,~L3A] \\
    R/\ab{c,t,c,t} &= ( ( (R / \ab{c}) / \ab{t} ) / \ab{c} ) / \ab{t} & 3 \times
    [1.8.3,~L2] \\
    &= c \ra t \ra R & 4 \times [1.8.3,~L3A] \\
    T2 / \ab{ } &= T2 = p \ra T2 & [1.8.3,~L1]
\end{align*}
The next step of this question is to show for each of the events $x \in
\sN{c,p,s,t}$ that $\ab{x} \not \in traces(Sy)$.
\begin{align*}
    traces(Sy) &= traces(\underbrace{((V/\ab{c,c})
                \cc (R/\ab{c,t,c,t}))}_{Sy'}
                \cc (T2/\ab{ }))
                & \\
    traces(Sy') &= \sN{t \ch
                    (t \filter \alpha V) \in traces(V / \ab{c,c}) \land
                    (t \filter \alpha R) \in traces(R / \ab{c,t,c,t}) \land
                    t \in (\alpha V \cup \alpha R)^*
                } & [2.3.3,~L1] \\
    traces(Sy) &= \sN{ t \ch
                    (t \filter \alpha Sy') \in traces(Sy') \land
                    (t \filter \alpha T2) \in traces(T2) \land
                    t \in (\alpha Sy' \cup \alpha R)^*
                  } & [2.3.3,~L1]
\end{align*}
When prooving that none of the singleton traces $\ab{x}$ for $x \in
\sN{c,p,s,t}$ are part of the trace of $Sy$, we need to derrive the set of
traces with a length of exactly one to leave out the possiblity of $\ab{x}$
being a part of the traces. These sets of traces are:
\begin{align*}
    traces(V/\ab{c,c}) &= \sN{\ab{ }} \cup
        \sN{\ab{s} \cat t \ch traces(V/\ab{c,c,s})} & [1.8.1,~L2] \\
    traces(R/\ab{c,t,c,t}) &= \sN{\ab{ }} \cup
    \sN{\ab{c} \cat t \ch traces(R/\ab{c,t,c,t,c})} & [1.8.1,~L2] \\
    traces(T2) &= \sN{\ab{ }} \cup
    \sN{\ab{p} \cat t \ch traces(T2/\ab{p})} & [1.8.1,~L2]
\end{align*}
After these expansions of the traces, the traces will only become longer and
hence we stop the trace ``search'' here.

For $x = c$ we have:
\begin{align*}
    \ab{c} \filter \alpha V &= \ab{c},~~
    \ab{c} \filter \alpha R = \ab{c},~~
    \ab{c} \filter \alpha T2 = \ab{ }
\end{align*}
Since $\ab{c} \not \in traces(V/\ab{c,c})$, we have shown that $\ab{c} \not \in
traces(Sy')$ and hence $\ab{c} \not \in traces(Sy)$.

For $p$ and $s$ we have:
\begin{align*}
    \ab{p} \filter \alpha V = \ab{p},~~
    \ab{p} \filter \alpha R = \ab{ },~~
    \ab{p} \filter \alpha T2 = \ab{p} \\
    \ab{s} \filter \alpha V = \ab{p},~~
    \ab{s} \filter \alpha R = \ab{ },~~
    \ab{s} \filter \alpha T2 = \ab{p}
\end{align*}
Since $\ab{p} \not \in traces(V/\ab{c,c})$, we have the same argument as above
and hence $\ab{p} \not \in traces(Sy)$ and $\ab{s} \not \in traces(Sy)$.

Looking at $t$:
\begin{align*}
    \ab{t} \filter \alpha V = \ab{ },~~
    \ab{t} \filter \alpha R = \ab{t},~~
    \ab{t} \filter \alpha T2 = \ab{ }
\end{align*}
We here have that $\ab{t} \not \in traces(R/\ab{c,t,c,t})$ and hence $\ab{t}
\not \in traces(Sy')$ and therefore $\ab{t} \not \in traces(Sy)$ and thus we
have completed the question.


\end{document}
