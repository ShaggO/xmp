\documentclass[11pt,a4paper]{article}

\usepackage[utf8]{inputenc}
\usepackage[english]{babel}
\usepackage[T1]{fontenc}
\usepackage{lmodern}
\usepackage[
    top     =   1in,
    right   = 0.5in,
    bottom  =   1in,
    left    = 0.5in]{geometry}

\usepackage{amsmath,amssymb,amsfonts}

\def\ra{\rightarrow}
\def\cc{\,\|\,}
\def\gc{\,\square\,}
\def\ch{\,|\,}
\def\ic{\,\sqcap\,}
\def\co{\backslash}
\def\cat{^{\frown}}
\def\count{\,\downarrow\,}
\def\filter{\,\upharpoonright\,}
\def\sat{\,\textbf{sat}\,}
\def\hide{\,\setminus\,}
\def\wr{\,!\,}
\def\re{\,?\,}
\def\after{\,/\,}
\newcommand{\chan}[1]{\textbf{#1}}
\newcommand{\ab}[1]{\left \langle #1 \right \rangle}
\newcommand{\sN}[1]{\left \lbrace #1 \right \rbrace}
\newcommand{\abs}[1]{\left | #1 \right |}

\renewcommand{\thesubsection}{\thesection.\alph{subsection}}

\title{Extreme Multiprogramming\\Assignment 2 - theory}
\author{Malte Stær Nissen\\ \texttt{tgq958}}

\begin{document}
\maketitle

\section{Equivalent definition of $NET$}

The assignment is to expand the following definition of $NET$ until it doesn't
contain any concealment, general choice or concurrent composition:
\begin{align*}
    STORE_x &\overset{\Delta}{=}(\chan{add} \re y \ra STORE_{x+y})
                                 \gc (\chan{fetch} \wr x \ra STORE_{x-1}) \\
    NET &=(STORE_{10} \cc (\chan{add} \wr 3 \ra STOP_A)) \hide A
\end{align*}
We first introduce the following auxilary processes $FETCH_x$ and $NETA_x$:
\begin{align*}
 FETCH_x &= STORE_x \cc STOP_A \\
 &= ( (\chan{add} \re y \ra STORE_{x+y}) \gc (\chan{fetch} \wr x \ra
 STORE_{x-1}) ) \cc STOP_A
            \\
        A &= \sN{\abs{\chan{add}}},~~ B = \sN{\chan{fetch}\wr x},~~ A \cap B =
        \sN{ } & [3.3.1,~L5] \\
 FETCH_x &= ( (\chan{add} \re y \ra STORE_{x+y} \ch \chan{fetch} \wr x \ra
 STORE_{x-1}) ) \cc STOP_A
            \\
            A &= \sN{\abs{\chan{add}}, \chan{fetch} \wr x},~~ B = \sN{ },~~ C
        = \sN{\chan{fetch} \wr x} & [2.3.1,~L7] \\
 FETCH_x &= \chan{fetch} \wr x \ra (STORE_{x-1} \cc STOP_A) \\
           &= \chan{fetch} \wr x \ra FETCH_{x-1} \\ \\
  NETA_x &= STORE_x \cc (\chan{add} \wr 3 \ra STOP_A) \\
         &= ((\chan{add} \re y \ra STORE_{x+y}) \gc (\chan{fetch} \wr x \ra
           STORE_{x-1})) \cc (\chan{add} \wr 3 \ra STOP_A) & \\
       A &= \sN{\abs{\chan{add}}},~~ B = \sN{\chan{fetch} \wr x},
       ~~ A \cap B = \sN{ } & [3.3.1,~L5] \\
       NETA_x &= (\chan{add} \re y \ra STORE_{x+y} \ch \chan{fetch} \wr x \ra
       STORE_{x-1}) \cc (\chan{add} \wr 3 \ra STOP_A) & \\
       A &= \sN{\abs{\chan{add}}, \chan{fetch}\wr x},~~ B = \sN{\chan{add}\wr3},~~
       C = \sN{\chan{add}.3, \chan{fetch} \wr x} & [2.3.1,~L7] \\
  NETA_x &= (\chan{add}.3 \ra (STORE_{x+3} \cc STOP_A) \ch
             \chan{fetch} \wr x \ra (STORE_{x-1} \cc (\chan{add} \wr 3 \ra
             STOP_A))) \\
         &= (\chan{add}.3 \ra FETCH_{x+3} \ch
         \chan{fetch} \wr x \ra NETA_{x-1})
\end{align*}
Which we now use the two defined processes above to define a process $NET_x$
similar to $NET$ but taking a start variable $x \in \mathbb{Z}$ as input:
\begin{align*}
    NET_x &=(STORE_x \cc (\chan{add} \wr 3 \ra STOP_A)) \hide A \\
        &= NETA_x \hide A \\
        &= (\chan{add}.3 \ra FETCH_{x+3} \ch \chan{fetch}\wr x \ra NETA_{x-1})
        \hide A \\
        C &= \sN{A},~~ B = \sN{\chan{add}.3, \chan{fetch}\wr x},~~
        C \cap B = \sN{\chan{add}.3},~~B - C = \sN{\chan{fetch}\wr x}
            & [3.5.1,~L10] \\
    NET_x &= (\underbrace{FETCH_{x+3} \hide A}_{FETCH'_{x+3}}) \ic
            ((FETCH_{x+3} \hide A) \gc
            (\chan{fetch}\wr x \ra (NETA_{x-1} \hide A)
            )) & \\
    FETCH'_{x} &= FETCH_x \hide A \\
               &= (\chan{fetch}\wr x \ra FETCH_{x-1}) \hide A \\
               &= \chan{fetch} \wr x \ra (FETCH_{x-1} \hide A) & [3.5.1,~L8] \\
               &= \chan{fetch} \wr x \ra FETCH'_{x-1} \\
    NET_x &= FETCH'_{x+3} \ic
             (FETCH'_{x+3} \gc
              (\chan{fetch}\wr x \ra NET_{x-1}
             )) & \\
          &= FETCH'_{x+3} \ic
             ((\chan{fetch} \wr (x+3) \ra FETCH'_{x+2}) \gc
              (\chan{fetch}\wr x \ra NET_{x-1}
             )) & \\
    A &= \sN{\chan{fetch} \wr (x+3)},~~ B = \sN{\chan{fetch} \wr x} &
            [3.3.1,~L5] \\
    NET_x &= FETCH'_{x+3} \ic
             (\chan{fetch} \wr (x+3) \ra FETCH'_{x+2} \ch
              \chan{fetch} \wr x \ra NET_{x-1}
             )
\end{align*}
This process is directly applicable to the $NET$ definition:
\begin{align*}
    NET &= NET_{10} \\
        &= FETCH'_{13} \ic (\chan{fetch}\wr 13 \ra FETCH'_{12} \ch
            \chan{fetch}\wr 10 \ra NET_{9})
\end{align*}


\section{Trace assignments}
\begin{align*}
    s = \ab{ \chan{fetch}.10, \chan{fetch}.12, \chan{fetch}.11}
\end{align*}

\subsection{$s \in traces(NET)$}

\subsection{$NET \after s$}

\section{Refusals of $NET$ and $NET \after s$}

\subsection{Refusals of $NET$}

\subsection{Refusals of $NET \after s$}

\end{document}

