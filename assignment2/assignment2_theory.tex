\documentclass[11pt,a4paper]{article}

\usepackage[utf8]{inputenc}
\usepackage[english]{babel}
\usepackage[T1]{fontenc}
\usepackage{lmodern}
\usepackage[
    top     =   1in,
    right   = 0.05in,
    bottom  =   1in,
    left    = 0.05in]{geometry}

\usepackage{amsmath,amssymb,amsfonts}

\def\ra{\rightarrow}
\def\cc{\,\|\,}
\def\gc{\,\square\,}
\def\ch{\,|\,}
\def\ic{\,\sqcap\,}
\def\co{\backslash}
\def\cat{^{\frown}}
\def\count{\,\downarrow\,}
\def\filter{\,\upharpoonright\,}
\def\sat{\,\textbf{sat}\,}
\def\hide{\,\setminus\,}
\def\wr{\,!\,}
\def\re{\,?\,}
\def\after{\,/\,}
\newcommand{\refusals}[1]{\text{refusals}\left ( #1 \right )}
\newcommand{\chan}[1]{\textbf{#1}}
\newcommand{\ab}[1]{\left \langle #1 \right \rangle}
\newcommand{\sN}[1]{\left \lbrace #1 \right \rbrace}
\newcommand{\abs}[1]{\left | #1 \right |}

\renewcommand{\thesubsection}{\thesection.\alph{subsection}}

\title{Extreme Multiprogramming\\Assignment 2 - theory}
\author{Malte Stær Nissen\\ \texttt{tgq958}}

\begin{document}
\maketitle

\section{Initial notes}

\section{Equivalent definition of $NET$}

The assignment is to expand the following definition of $NET$ until it doesn't
contain any concealment, general choice or concurrent composition:
\begin{align*}
    STORE_x &\overset{\Delta}{=}(\chan{add} \re y \ra STORE_{x+y})
                                 \gc (\chan{fetch} \wr x \ra STORE_{x-1}) \\
    NET &=(STORE_{10} \cc (\chan{add} \wr 3 \ra STOP_A)) \hide A
\end{align*}
We first introduce the following auxilary processes $FETCH_x$ and $NETA_x$:
\begin{align*}
 FETCH_x &= STORE_x \cc STOP_A \\
 &= ( (\chan{add} \re y \ra STORE_{x+y}) \gc (\chan{fetch} \wr x \ra
 STORE_{x-1}) ) \cc STOP_A
            \\
        A &= \sN{\abs{\chan{add}}},~~ B = \sN{\chan{fetch}\wr x},~~ A \cap B =
        \sN{ } & [3.3.1,~L5] \\
 FETCH_x &= ( (\chan{add} \re y \ra STORE_{x+y} \ch \chan{fetch} \wr x \ra
 STORE_{x-1}) ) \cc STOP_A
            \\
            A &= \sN{\abs{\chan{add}}, \chan{fetch} \wr x},~~ B = \sN{ },~~ C
        = \sN{\chan{fetch} \wr x} & [2.3.1,~L7] \\
 FETCH_x &= \chan{fetch} \wr x \ra (STORE_{x-1} \cc STOP_A) \\
           &= \chan{fetch} \wr x \ra FETCH_{x-1} \\ \\
  NETA_x &= STORE_x \cc (\chan{add} \wr 3 \ra STOP_A) \\
         &= ((\chan{add} \re y \ra STORE_{x+y}) \gc (\chan{fetch} \wr x \ra
           STORE_{x-1})) \cc (\chan{add} \wr 3 \ra STOP_A) & \\
       A &= \sN{\abs{\chan{add}}},~~ B = \sN{\chan{fetch} \wr x},
       ~~ A \cap B = \sN{ } & [3.3.1,~L5] \\
       NETA_x &= (\chan{add} \re y \ra STORE_{x+y} \ch \chan{fetch} \wr x \ra
       STORE_{x-1}) \cc (\chan{add} \wr 3 \ra STOP_A) & \\
       A &= \sN{\abs{\chan{add}}, \chan{fetch}\wr x},~~ B = \sN{\chan{add}\wr3},~~
       C = \sN{\chan{add}.3, \chan{fetch} \wr x} & [2.3.1,~L7] \\
  NETA_x &= (\chan{add}.3 \ra (STORE_{x+3} \cc STOP_A) \ch
             \chan{fetch} \wr x \ra (STORE_{x-1} \cc (\chan{add} \wr 3 \ra
             STOP_A))) \\
         &= (\chan{add}.3 \ra FETCH_{x+3} \ch
         \chan{fetch} \wr x \ra NETA_{x-1})
\end{align*}
Which we now use the two defined processes above to define a process $NET_x$
similar to $NET$ but taking a start variable $x \in \mathbb{Z}$ as input:
\begin{align*}
    NET_x &=(STORE_x \cc (\chan{add} \wr 3 \ra STOP_A)) \hide A \\
        &= NETA_x \hide A \\
        &= (\chan{add}.3 \ra FETCH_{x+3} \ch \chan{fetch}\wr x \ra NETA_{x-1})
        \hide A \\
        C &= \sN{A},~~ B = \sN{\chan{add}.3, \chan{fetch}\wr x},~~
        C \cap B = \sN{\chan{add}.3},~~B - C = \sN{\chan{fetch}\wr x}
            & [3.5.1,~L10] \\
    NET_x &= (\underbrace{FETCH_{x+3} \hide A}_{FETCH'_{x+3}}) \ic
            ((FETCH_{x+3} \hide A) \gc
            (\chan{fetch}\wr x \ra (NETA_{x-1} \hide A)
            )) & \\
    FETCH'_{x} &= FETCH_x \hide A \\
               &= (\chan{fetch}\wr x \ra FETCH_{x-1}) \hide A \\
               &= \chan{fetch} \wr x \ra (FETCH_{x-1} \hide A) & [3.5.1,~L8] \\
               &= \chan{fetch} \wr x \ra FETCH'_{x-1} \\
    NET_x &= FETCH'_{x+3} \ic
             (FETCH'_{x+3} \gc
              (\chan{fetch}\wr x \ra NET_{x-1}
             )) & \\
          &= FETCH'_{x+3} \ic
             ((\chan{fetch} \wr (x+3) \ra FETCH'_{x+2}) \gc
              (\chan{fetch}\wr x \ra NET_{x-1}
             )) & \\
    A &= \sN{\chan{fetch} \wr (x+3)},~~ B = \sN{\chan{fetch} \wr x} &
            [3.3.1,~L5] \\
    NET_x &= FETCH'_{x+3} \ic
             \underbrace{(\chan{fetch} \wr (x+3) \ra FETCH'_{x+2} \ch
             \chan{fetch} \wr x \ra NET_{x-1})}_{NET'_x}
\end{align*}
This process is directly applicable to the $NET$ definition:
\begin{align*}
    NET &= NET_{10} \\
        &= FETCH'_{13} \ic (\chan{fetch}\wr 13 \ra FETCH'_{12} \ch
            \chan{fetch}\wr 10 \ra NET_{9})
\end{align*}


\section{Trace assignments}
We introduce the following trace $s$:
\begin{align*}
    s = \ab{ \chan{fetch}.10, \chan{fetch}.12, \chan{fetch}.11}
\end{align*}

\subsection{$s \in traces(NET)$}
To show that $s \in traces(NET)$, we will derrive the trace $u = s$ from the
definition of $NET$ by letting $s$ guide the derrivations. The laws stated in
the equations below are used for the reasoning in each line. When reasoning
about $FETCH'_x$ we adhere from explicitly substituting $FETCH'_x$ by it's
definition: $\chan{fetch}\wr x \ra FETCH'_{x-1}$ to shorten the lines:
\begin{align*}
    &s \in traces(NET) ~\text{if:} \\
    &\hspace{1cm}s \in traces(FETCH'_{13}) ~\text{or} \\
    &\hspace{1cm}s \in traces(
    (\chan{fetch}\wr 13 \ra FETCH'_{12} \ch \chan{fetch} \wr 10 \ra NET_9)) &
[3.2.3,~L1] \\
&s \in traces((\chan{fetch}\wr 13 \ra FETCH'_{12} \ch \chan{fetch} \wr 10 \ra
NET_9))~\text{if:} \\
    &\hspace{1cm}\ab{\chan{fetch}.12, \chan{fetch}.11} \in
    traces(NET_9) & [1.8.1,~L3] \\
    &\ab{\chan{fetch}.12,\chan{fetch}.11} \in traces(NET_9) ~\text{if:} \\
    &\hspace{1cm}\ab{\chan{fetch}.12,\chan{fetch}.11} \in traces(FETCH'_{12}) ~\text{or} \\
    &\hspace{1cm}\ab{\chan{fetch}.12,\chan{fetch}.11} \in
    traces((\chan{fetch}\wr 12 \ra FETCH'_{11} \ch \chan{fetch}\wr 9 \ra NET_8))
    & [3.2.3,~L1] \\
    &\ab{\chan{fetch}.12,\chan{fetch}.11} \in traces(FETCH'_{12}) ~\text{if:}\\
    &\hspace{1cm}\ab{\chan{fetch}.11} \in traces(FETCH'_{11}) & [1.8.1,~L3] \\
    &traces(FETCH'_{11}) = \sN{\ab{ }} \cup 
    \sN{\ab{\chan{fetch}.11} \cat t \ch t \in traces(FETCH'_{10})}\\
    &\hspace{1cm} \Rightarrow \ab{\chan{fetch}.11} \in traces(FETCH'_{11})
    & [1.8.1,~L2] \\
    &\hspace{1cm} \Rightarrow s \in traces(NET)
\end{align*}

\subsection{$NET \after s$}
NOTE: When computing $FETCH'_x \after \ab{\chan{fetch}\wr x}$ where $x \in
\mathbb{Z}$ I will adhere from explicitly expanding $FETCH'_x$ to $\chan{fetch}\wr
x \ra FETCH'_{x-1}$ before reducing to $FETCH'_{x-1}$ in the following step to
avoid cluttering the derrivations with this particular trivial substitution.
\begin{align*}
    NET \after s &= (FETCH'_{13} \ic
                    (\chan{fetch}\wr 13 \ra FETCH'_{12} \ch
                     \chan{fetch}\wr 10 \ra NET_{9})) \after s \\
                 &= (FETCH'_{13} \ic
                     (\chan{fetch}\wr 13 \ra FETCH'_{12} \ch
                      \chan{fetch}\wr 10 \ra NET_{9})) \after
                      \ab{\chan{fetch}.10, \chan{fetch}.12, \chan{fetch}.11} \\
                 &= ((FETCH'_{13} \ic
                     (\chan{fetch}\wr 13 \ra FETCH'_{12} \ch
                     \chan{fetch}\wr 10 \ra NET_{9})) \after
                     \ab{\chan{fetch}.10} ) \after
                     \ab{\chan{fetch}.12, \chan{fetch}.11} & [1.8.3,~L2]
\end{align*}
We now use law $[3.2.3,~L2]$ where we first specify the traces of the sets $P$
and $Q$ in the internal choice:
\begin{align*}
  traces(P) &= traces(FETCH'_{13}) \\
            &= \sN{\ab{ }} \cup
                \sN{\ab{\chan{fetch}.13} \cat t \ch t \in traces(FETCH'_{12})}
                & [1.8.1,~L2] \\
  traces(Q) &= traces(\chan{fetch}\wr 13 \ra FETCH'_{12} \ch
                                    \chan{fetch}\wr 10 \ra NET_{9}) \\
            &= \sN{t \ch t = \ab{ } \lor
            (t_0 = \chan{fetch}\wr 13 \land t' \in traces(FETCH'_{12})) \lor
            (t_0 = \chan{fetch}\wr 10 \land t' \in traces(NET_9))
            } & [1.8.1,~L3] \\
            &= \sN{\ab{ }} \cup
                    \sN{\ab{\chan{fetch}.13} \cat t \ch t \in traces(FETCH'_{12})}
                    \cup
                    \sN{\ab{\chan{fetch}.10} \cat t \ch t \in traces(NET_9)}
\end{align*}
Since $\ab{\chan{fetch}.10} \in (traces(Q) - traces(P))$ we conclude:
\begin{align*}
    NET \after s &= ((\chan{fetch}\wr 13 \ra FETCH'_{12} \ch
    \chan{fetch}\wr 10 \ra NET_{9}) \after \ab{\chan{fetch}.10} ) \after
    \ab{\chan{fetch}.12, \chan{fetch}.11} & [3.2.3,~L2] \\
                 &= NET_9 \after \ab{\chan{fetch}.12,\chan{fetch}.11} &
                 [1.8.3,~L3A] \\
                 &= (FETCH'_{12} \ic
                    (\chan{fetch}\wr 12 \ra FETCH'_{11} \ch
                     \chan{fetch}\wr 9 \ra NET_{8})) \after
                     \ab{\chan{fetch}.12,\chan{fetch}.11} \\
                 &= ((FETCH'_{12} \ic
                    (\chan{fetch}\wr 12 \ra FETCH'_{11} \ch
                    \chan{fetch}\wr 9 \ra NET_{8})) \after \ab{\chan{fetch}.12})
                    \after \ab{\chan{fetch}.11} & [1.8.3,~L2]
\end{align*}
Again we compute the traces of the sets $P$ and $Q$ for law $[3.2.3,~L2]$:
\begin{align*}
    traces(P) &= traces(FETCH'_{12}) \\
              &= \sN{\ab{ }} \cup
              \sN{\ab{\chan{fetch}.12} \cat t \ch t \in traces(FETCH'_{11}} &
              [1.8.1,~L2] \\
    traces(Q) &= traces(\chan{fetch}\wr 12 \ra FETCH'_{11} \ch
                        \chan{fetch}\wr 9  \ra NET_{8}) \\
              &= \sN{\ab{ }} \cup
              \sN{\ab{\chan{fetch}.12 \cat t \ch t \in traces(FETCH'_{11})}}
              \cup
              \sN{\ab{\chan{fetch}.9 \cat t \ch t \in traces(NAT_{8})}}
\end{align*}
This time we see that $\chan{fetch}.12 \in (traces(P) \cap traces(Q))$ and hence
by $[3.2.3,~L2]$ we get:
\begin{align*}
    NET \after s &= ((FETCH'_{12} \after \ab{\chan{fetch}.12}) \ic
                    ((\chan{fetch}\wr 12 \ra FETCH'_{11} \ch
                    \chan{fetch}\wr 9 \ra NET_{8}) \after \ab{\chan{fetch}.12})) \after
                    \ab{\chan{fetch}.11}
\end{align*}
\begin{align*}
    NET \after s &= (FETCH'_{11}\ic
                     FETCH'_{11}) \after
                    \ab{\chan{fetch}.11} & 2 \times [1.8.3,~L3A] \\
                 &= FETCH'_{11} \after \ab{\chan{fetch}.11} & [3.2.1,~L1] \\
                 &= FETCH'_{10} & [1.8.3,~L3A] \\
\end{align*}


\section{Refusals of $NET$ and $NET \after s$}

\subsection{Refusals of $NET$}
Remember the definition of $NET_x$ and $NET$:
\begin{align*}
    NET_x &= FETCH'_{x+3} \ic
             \underbrace{(\chan{fetch} \wr (x+3) \ra FETCH'_{x+2} \ch
             \chan{fetch} \wr x \ra NET_{x-1})}_{NET'_x} \\
    NET &= NET_{10} = FETCH'_{13} \ic NET'_{10}
\end{align*}
And need to show that:
\begin{align*}
    \sN{\chan{fetch}.10, \chan{fetch}.42} \in \refusals{NET} \\
    \sN{\chan{fetch}.13} \not \in \refusals{NET}
\end{align*}

We compute the refusals of $NET$:
\begin{align*}
    \refusals{NET} &=
    \refusals{FETCH'_{13}} \cup \refusals{NET'_{10}} & [3.4.1,~L4] \\
    \refusals{FETCH'_{13}} &=
    \refusals{ X \ch X \subseteq (\alpha(FETCH'_{12}) - \sN{\chan{fetch}\wr 13}) }
        & [3.4.1,~L2] \\
    B &= \sN{\chan{fetch}\wr 13,\chan{fetch}\wr 10} & [3.4.1,~L3] \\
    \refusals{NET_{10}} &= \sN{ X \ch X \subseteq (\alpha(NET_{10}) - 
            \sN{\chan{fetch}\wr 13,\chan{fetch}\wr 10})}
\end{align*}
Looking at these urefusals we can quickly conclude that $\sN{\chan{fetch}.10,
\chan{fetch}.42} \in \refusals{NET}$ since the only 1-length-elements subtracted from
$\sN{\abs{\chan{fetch}}}$ are $\chan{fetch}\wr 13$ and $\chan{fetch}\wr 10$.
This is likewise the argument for concluding that $\sN{\chan{fetch}.13} \not \in
traces(NET)$.

\subsection{Refusals of $NET \after s$}
Remember the definition of $NET \after s$:
\begin{align*}
    NET \after s &= FETCH'_{10}
\end{align*}
And need to show that:
\begin{align*}
    \sN{\chan{fetch}.13, \chan{fetch}.42} \in \refusals{NET \after s} \\
    \sN{\chan{fetch}.10} \not \in \refusals{NET \after s}
\end{align*}

We compute the 1-size-elements of the refusals:
\begin{align*}
    \refusals{NET \after s} &= \refusals{\chan{fetch}\wr 10 \ra FETCH'_9} \\
    &= \sN{ X \ch X \subseteq (\alpha(FETCH'_9) - \sN{\chan{fetch}\wr 10}}
        & [3.4.1,~L2]
\end{align*}
Again we use the same argument as in the exercise above. The only 1-size-element
from the set $\sN{\abs{\chan{fetch}}}$ subtracted from the refusal set is
$\chan{fetch}\wr 10$ and hence $\sN{\chan{fetch}.13,\chan{fetch}.42} \in
traces(NET \after s)$ and $\sN{\chan{fetch}.10} \not \in traces(NET \after s)$.
\end{document}
