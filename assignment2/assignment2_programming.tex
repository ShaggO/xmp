\documentclass[11pt,a4paper]{article}

\usepackage[utf8]{inputenc}
\usepackage[english]{babel}
\usepackage[T1]{fontenc}
\usepackage{lmodern}
\usepackage{minted}
\usepackage{amsmath,amssymb,amsfonts}

\title{Extreme Multiprogramming\\Assignment 2 - programming}
\author{Malte Stær Nissen}

\begin{document}
\date{\today}

\maketitle
\section{Considerations on concurrency in the game}
The system is set up to let the room processes perform all data-handling.
Each client has a corresponding user object with a reading end of a channel
for reading user commands and a writing end for another channel returning
data/control to the client once the room has processed the given command.
Furthermore each client has a notification channel and a corresponding
notification reader process which receives notifications of other clients
entering and leaving the room which the client is currently in. The
notification retrieval is performed in an asynchronous process to avoid
deadlocks when notifying people in other rooms.

Each room has an ``input'' channel (reading end), which all the adjacent
rooms can write to in order to transfer a client into the room. The actual
transfer is performed in an asynchronous process in order to avoid deadlocks
it two rooms start transferring players to one another at the same time. A
transfer is performed by first removing the player ``object'' from the room
it resides and then write it to the input channel of the new room followed by
notifications to all players in question and a final return signal (True) to
the client return channel.

\section{Shutdown of the system}
Since each room has a list of clients and client notification processes as
well as channels to the adjacent rooms, a shutdown is a simple poison of all
writing endpoints the room receiving an exit command. This will eventually
force all rooms to poison all their clients and thus shutting down the entire
network.

\section{Graphical CSP-network design}

\clearpage
\appendix

\section{room.py}
\inputminted[linenos,fontsize=\scriptsize]{python}{src/room.py}

\section{agent.py}
\inputminted[linenos,fontsize=\scriptsize]{python}{src/agent.py}

\section{player.py}
\inputminted[linenos,fontsize=\scriptsize]{python}{src/player.py}

\section{game.py}
\inputminted[linenos,fontsize=\scriptsize]{python}{src/game.py}

\end{document}

